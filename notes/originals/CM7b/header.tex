
\documentclass[11pt]{scrartcl} %Type de doc. %scrartcl
\usepackage[french,english]{babel}
\usepackage[utf8]{inputenc} %Caracteres accentués           
\usepackage[T1]{fontenc} 	%Police accentuée
\usepackage{lmodern}			%Police vectorielle (haute qualité)
%\usepackage{vmargin} %Marges normales en A4
\usepackage{amsmath} %Insertion d'equations
\usepackage{todonotes}

\usepackage{tikz}
\usetikzlibrary{shapes,arrows}

\usepackage{amssymb}
\usepackage{vmargin} 

\usepackage{graphicx} %Images dans le PDF
\usepackage{float} 		%Flottants : Figures,Tables
\usepackage{color}		%Utilisation de couleurs
\definecolor{gris}{gray}{0.5}
\usepackage{eurosym}
\usepackage[hyphens]{url}
\usepackage{hyperref}
% \usepackage{breakurl}
\usepackage{wrapfig}
\usepackage{multirow} 
\usepackage{rotating}
\usepackage[small,bf]{caption}
\usepackage{textcomp}

\usepackage{subfig} 

\usepackage{amsthm}
\usepackage{pgf}

\usepackage{epstopdf}
\usepackage{attachfile} 
\usepackage{longtable}

\renewcommand*{\thefootnote}{$\langle\arabic{footnote}\rangle$}

\theoremstyle{definition}
\newtheorem{example}{{\bf Example}}[part]
\newtheorem{theorem}[example]{{\bf Theorem}}
\newtheorem{property}[example]{{\bf Property}}
\newtheorem{definition}[example]{{\bf Definition}}
\newtheorem{remark}[example]{{\bf Remark}}
\newtheorem{remarks}[example]{{\bf Remarks}}
\newtheorem{lemma}[example]{{\bf Lemma}}
\newtheorem{hypothesis}[example]{{\bf Hypothesis}}
\newtheorem{exercise}{{\bf Exercise}}[part]

\setkomafont{part}{\LARGE}
\setkomafont{section}{\Large}
\setkomafont{subsection}{\large}
\setkomafont{subsubsection}{\normalsize}

\setcounter{secnumdepth}{3}
\setcounter{tocdepth}{3}

\allowdisplaybreaks[3] % ou x prend une valeur entière comprise entre 0 et 4; Plus x est gd, plus le compilateur accepte les coupures sur deux pages dans une formule.
% permet que les align puissent se prolonger sur pls pages

%========================================
\usepackage{listings}	%Inclusion de code-source
\lstset{							%Paramètres généraux pour les inclusions de code
	flexiblecolumns=true,
	numbers=left,
	numberstyle=\ttfamily\tiny,
	keywordstyle=\textcolor{blue},
	stringstyle=\textcolor{red},
	commentstyle=\textcolor{green},
	breaklines=true,
	extendedchars=true,
	basicstyle=\ttfamily\scriptsize,
	showstringspaces=false,
	captionpos=b	
	}
\renewcommand{\lstlistingname}{\textsc{Code}}	%Remplacer 'Listing' par 'Code' dans les légendes
%\usepackage[french,boxed,linesnumbered]{algorithm2e}	%Package algorithme avec options
\usepackage[english,algoruled, linesnumbered]{algorithm2e}	%Package algorithme avec options

\lstset{inputencoding=Latin1}

\hypersetup{
colorlinks,%
citecolor=black,%
filecolor=black,%
linkcolor=black,%
urlcolor=black
} 

\makeatletter
\def\url@leostyle{%
  \@ifundefined{selectfont}{\def\UrlFont{\sf}}{\def\UrlFont{\small\ttfamily}}}
\makeatother
%% Now actually use the newly defined style.
\urlstyle{leo}


