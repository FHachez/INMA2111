\chapter{Random Algorithms}

\colorbox{green}{TEMPORARY - CM7}

First, let's do a few exercices in probability.

\section{The marriage/secretary/harem/... problem}
$n$ candidates of different quality apply for an open position. They are sent in random order. Every time you interview a candidate who proves better than any other previous one, you hire him/her.

\subsection{How many persons will you hire on average?}
Say you interview them 
\begin{itemize}
\item in increasing order $q_1 < q_2 < \ldots < q_n$ ($1,2, \ldots, n$ denote the order of the interviews) $\Rightarrow n$ hires;
\item in decreasing order $q_1 > q_2 > \ldots > q_n \Rightarrow 1$ hire;
\item in random order $\Rightarrow$ something in between.
\end{itemize}
To find the number of persons we will hire on average, we will use the powerful technique of \textit{indicator variable}. Let
\begin{align*}
X_i &=
  \left\{
      \begin{aligned}
      	1 &\text{ if person $i$ is hired} \\
        0 &\text{ it not }
      \end{aligned}
    \right.\\
\mathbb{E}[X_i] &= \text{P[person $i$ is hired}] &\text{ fondamental property of indicator variable}
\end{align*}
We want to find
\begin{align*}
\mathbb{E}[X] &=  \mathbb{E}[X_1+X_2+\ldots+X_n] & \\
&=  \mathbb{E}[X_1] + \mathbb{E}[X_2] + \ldots +\mathbb{E}[X_n]  &\text{since the expectation is linear}\\
&= \sum_{i=1}^n \text{P[$i$ is hired]} &\\
&= \sum_{i=1}^n \frac{1}{i} &\text{because the best among $1,2,\ldots,i$ is $i$ with probability $\frac{1}{i}$}\\
&\approxeq \int_1^n \frac{1}{i}  \, \mathrm{d}i. &\text{continuous problem}\\
&= \ln n. &
\end{align*}

\subsection{Variant}
This time, we hire only 1 person and we want to maximize the probability to pick the best among the $n$ candidates (as soon as we interview someone, we need to say yes or no).

\begin{lstlisting}[label={list:c7:hiring problem},caption=Pseudo-code of the Variant of Hiring Problem algorithm]
Don't hire the first k-1 candidates (observation phase).
Hire the first among the candidates k, k+1,..., n to be the best interviewed so far.
\end{lstlisting}

\subsection{What is the best $k$?}
\begin{align*}
\text{Probability of success} &= \sum_{i=k}^n \text{P[$i$ is hired and is the best]}\\
&= \sum_{i=k}^n  \text{[$i$ is the best]} \times \text{P[$i$ is hired | $i$ is the best]}\\
&= \sum_{i=k}^n  \frac{1}{n} \frac{k-1}{i-1}\\
&\approxeq \frac{k}{n} \int_{k-1}^{n-1} \frac{1}{x} \, \mathrm{d}x.\\
&= \frac{k}{n} (\ln n - \ln k)\\
&= \frac{\ln \frac{n}{k}}{\frac{n}{k}}\\
&= \frac{\ln y}{y} \text{for $y = \frac{n}{k} \geq 1$}
\end{align*}
To find this, we need to notice that if the second best among $1,\ldots,i$ is in 
\begin{itemize}
\item $1,\ldots, k-1$ then $i$ is picked;
\item $k,\ldots, i-1$ then $i$ is not picked.
\end{itemize}
So, $i$ is picked with probability $\frac{k-1}{i-1}$.

Then, $y$ is optimal for 
\begin{align*}
\frac{\mathrm{d}}{\mathrm{d}y} \left(\frac{\ln y}{y} \right) &= 0\\
\frac{1- \ln y}{y^2} &= 0\\
\Rightarrow \ln y &= 1\\
y &= e
\end{align*}
So $k = \frac{n}{e} \approxeq 0.37n$ is optimal and the corresponding probability of success is $\frac{\ln \frac{n}{k}}{\frac{n}{k}} = \frac{1}{e} \approxeq 0.37$.



\section{The birthday paradox}
How many people do you need to gather so that two of them will have the same birthday with fair chances? We will solve this problem with indicator variables.
\begin{align*}
X_{ij} &=
  \left\{
      \begin{aligned}
      	1 &\text{ if $i$ and $j$ have same birthday} \\
        0 &\text{ it not }
      \end{aligned}
    \right.\\
\mathbb{E}[X_{ij}] &= \frac{1}{t} \text{ where $t = 365$}\\
\sum_{pairs \{i,j\}} X_{i,j} &= X\\
&= \text{\# of coincidences}\\
&= \text{\# pairs of people with same birthday}\\
\mathbb{E}[X] &=  \mathbb{E}[\sum_{i,j} X_{i,j}] \\
&= \sum_{i,j}  \mathbb{E}[X_{i,j}] \\
&= \frac{n(n-1)}{2} \frac{1}{t} \approxeq \frac{n^2}{2t} \text{ for $n$ persons}
\end{align*}
So, if $n = \sqrt{2t}$, then $\mathbb{E}[X] \approxeq 1$, i.e. there exist a fair chance to observe a coincidence. If $t = 365$, $n = \sqrt{2*365} = 25$. Therefore already with $25$ people there is a very good chance that two people were born on the same day, whereas we would expect that more people would be needed.