\chapter{Divide and Conquer}

\colorbox{green}{TEMPORARY - CM3}

\section{Multiplication of 2 large integers}
How to multiply 2 large integers (of $n$ digits each)?\\
Elementary method : use the grid method for written multiplication :
\begin{center}
\begin{tikzpicture}
\draw(3,10)--(6,10);
\draw(3,10)--(3,10.5);
\draw(3,10.5)--(6,10.5);
\draw(6,10.5)--(6,10);

\draw(3,10.7)--(6,10.7);
\draw(3,10.7)--(3,11.2);
\draw(3,11.2)--(6,11.2);
\draw(6,11.2)--(6,10.7);

\node at (2.75,10.2) {$\times$};

\draw(1,9.75)--(8,9.75);

\draw(3,9)--(6,9);
\draw(3,9)--(3,9.5);
\draw(3,9.5)--(6,9.5);
\draw(6,9.5)--(6,9);

\draw(2.5,8.3)--(5.5,8.3);
\draw(2.5,8.3)--(2.5,8.8);
\draw(2.5,8.8)--(5.5,8.8);
\draw(5.5,8.8)--(5.5,8.3);

\draw(2,7.6)--(5,7.6);
\draw(2,7.6)--(2,8.1);
\draw(2,8.1)--(5,8.1);
\draw(5,8.1)--(5,7.6);

\draw(1.5,6.9)--(4.5,6.9);
\draw(1.5,6.9)--(1.5,7.4);
\draw(1.5,7.4)--(4.5,7.4);
\draw(4.5,7.4)--(4.5,6.9);

\node at (1.25,7.1) {$+$};

\draw(1,6.65)--(8,6.65);

\node at (4.5,6.4) {$\hdots\hdots\hdots$};

\node at (10,9) { : $\Theta(n^2)$};
\end{tikzpicture}
\end{center}

We can do better!\\
\begin{align*}
x\times y&=\big(\underbrace{x_0}_{n/2 \text{ digits}}+10^{\left\lfloor n/2\right\rfloor} \underbrace{x_1}_{n/2}\big)\times \big(\underbrace{y_0}_{n/2}+10^{\left\lfloor n/2\right\rfloor} \underbrace{y_1}_{n/2}\big)\\
&=x_0 \times y_0 + 10^{\left\lfloor n/2\right\rfloor} (x_1 \times y_0 + x_0 \times y_1)+10^{2{\left\lfloor n/2\right\rfloor}}(x_1\times y_1)
\end{align*}
\emph{Complexity} : $t_n=4t_{\left\lfloor n/2\right\rfloor}+\Theta(n)$\\
\emph{Solution} : if we assume $n\approx 2^k$, we have : 
\begin{align*}
t_n&=4t_{n/2}+\Theta(n)\\
&=16t_{n/4}+4 \Theta(n/2)+\Theta(n)\\
&=64t_{n/8}+\underbrace{16\Theta(n/4)}_{\Theta(4n)}+\underbrace{4 \Theta(n/2)}_{\Theta(2n)}+\underbrace{\Theta(n)}_{\Theta(n)}\\
&=\hdots\\
&=2^{2k}\underbrace{t_{n/2^k}}_{\mathcal{O}(1)}+\Theta(\underbrace{(2^{k-1}+2^{k-2}+\hdots+1)}_{2^k-1}n)=\Theta(n^2)
\end{align*}
This is not better.\\

\subsection{Master Theorem}

This kind of recurrence is solved by the \emph{Master Theorem}.\\
\begin{theorem}[Master]
\label{MasterTheorem}
For the recurrence $t_n=at_{n/b}+f(n)$ (or $t_{\lfloor n/b \rfloor}$ or $t_{\lceil n/b \rceil}$) : 
\begin{enumerate}% [a)]
\item If $f(n)\in \mathcal{O}\left(n^{\log_ba-\epsilon}\right)$ for some $\epsilon>0$, then $t_n \in \Theta\left(n^{\log_ba}\right)$; \label{a}
\item If $f(n)\in \Theta\left(n^{\log_ba}\right)$, then $t_n \in \Theta\left(n^{\log_ba}\log n\right)$; \label{b}
\item If $f(n)\in \Omega\left(n^{\log_ba+\epsilon}\right)$ for some $\epsilon>0$, $af(n/b)\leq cf(n)$ for some $c<1$ and $n$ is large enough , then $t_n \in \Theta\left( f(n)\right)$.
\end{enumerate}
\end{theorem}

\begin{example}
\begin{leftbar}
We can give the following examples
\begin{itemize}
\item[$\bullet$] $t_n=2t_{n/2}+\Theta(n)$ : $a=2$, $b=2$ and $\log_2 2=1$\\
By \ref{b} from Theorem \ref{MasterTheorem}, $f(n)\in \Theta\left(n^{\log_ba}\right)\rightarrow t_n \in \Theta(n\log n)$
\item $t_n=4t_{n/2}+\Theta(n)$: $a=4$, $b=2$ and $\log_2 4=2$\\
By \ref{a}, $f(n)=\Theta\left(n^{\log_2 4-1}\right)\rightarrow t_n=\Theta\left(n^{\log_ba}\right)=\Theta\left(n^2\right)$
\end{itemize}
\end{leftbar}
\end{example}


\subsection{Karastuba-Ofman}

It is possible to make only 3 multiplications of $\dfrac{n}{2}$-digits numbers.

\begin{lstlisting}[label={list:c3:Multiply(x,y)},caption=\texttt{Multiply(x,y)} \textmd{(Karastuba-Ofman, 1962)}]
x=x_0+(10^floor(n/2))*x_1
y=y_0+(10^floor(n/2))*y_1
M1=(x_0+y_0)*(x_1+y_1)
M2=x_0*y_0
M3=x_1*y_1
x*y=M2+10^(n/2)*(M1-M2-M3)+10^(2(n/2))*M3
\end{lstlisting}


\emph{Time complexity} : 
\begin{align*}
t_n=3t_{n/2}+\Theta (n) \Rightarrow t_n=\Theta\left(n^{\log_2 3}\right)\approx \Theta\left(n^{1.585}\right)
\end{align*}
We can do even better!$\rightarrow$ Schönhage-Strassen (1971) : $\mathcal{O}(n\log(n)\log\log(n))$.

\section{Multiplication of 2 matrices}

How can we multiply two $n$-by-$n$ matrices?\\
Elementary algorithm : $\Theta(n^3)$ elementary operations.

\subsection{Block Multiplication}
Is block multiplication better?
\begin{align*}
&A=\left(\begin{array}{c|c}
A_{11} & A_{12}\\
\hline
A_{21} & A_{22}
\end{array}\right) \hspace*{2cm} B=\left(\begin{array}{c|c}
B_{11} & B_{12}\\
\hline
B_{21} & B_{22}
\end{array}\right)\\
\rightarrow &AB=\begin{pmatrix}
A_{11}B_{11}+A_{12}B_{21} & A_{11}B_{12}+A_{12}B_{22}\\
A_{21}B_{11}+A_{22}B_{21} & A_{21}B_{12}+A_{22}B_{22}
\end{pmatrix}
\end{align*}
\emph{Time complexity} : 8 products of $\dfrac{n}{2}$-by-$\dfrac{n}{2}$ matrices.
\begin{align*}
t_n=8t_{n/2}+\Theta(n^2)=\Theta\left(n^{\log_2 8}\right)=\Theta\left(n^3\right)
\end{align*}
This is not better. 

\subsection{Strassen's Algorithm}
We can do it with only 7 products (Strassen's algorithm, 1969) : 
\begin{align*}
&\left\{
\begin{array}{l}
M_1=(A_{21}+A_{22}-A_{11})(B_{22}-B_{12}+B_{11})\\
M_2=A_{11}B_{11}\\
M_3=A_{12}B_{21}\\
M_4=(A_{11}-A_{21})(B_{22}-B_{12})\\
M_5=(A_{21}+A_{22})(B_{12}-B_{11})\\
M_6=(A_{12}-A_{21}+A_{11}-A_{22})B_{22}\\
M_7=A_{22}
\end{array}
\right.
\end{align*}
\begin{align*}
\rightarrow & AB=\begin{pmatrix}
M_2+M_3 & M_1+M_2+M_5+M_6\\
M_1+M_2+M_4-M_7 & M_1+M_2+M_4+M_5
\end{pmatrix}
\end{align*}

\emph{Time complexity} : $t_n=7t_{n/2}+\Theta(n^2)=\Theta\left(n^{\log_2 7}\right)=\Theta\left(n^{2.81}\right)$\\
We can do even better!$\rightarrow$ Coppersmith-Winograd (1981) : $\Theta(n^{2.376})$\\
Best one? Unknown but $\Omega(n^2)$.

\subsection{Matrix Inversion}
What is the cost of matrix inversion?\\
Elementary methods : $\Theta(n^3)$
\begin{theorem}
Matrix inversion and multiplication have the same complexity
\end{theorem}
\begin{proof}
Let $I(n)$ be the (worst-case) complexity of inversion (i.e. of the best possible algorithm) and let $M(n)$ be the (worst-case) complexity of multiplication (i.e. of the best possible algorithm).\\
We want to show that $I(n)\in \Theta(M(n))$.
\begin{enumerate}% [a)]
\item $M(n)\in \mathcal{O}(I(n))$?\\
We reduce multiplication to inversion.\\

$$D=\underbrace{\begin{pmatrix}
I_n & A & 0\\
0 & I_n & B\\
0 & 0 & I_n
\end{pmatrix}}\limits_{3n*3n}
\rightarrow D^{-1}=\begin{pmatrix}
I_n & -A & AB\\
0 & I_n & -B\\
0 & 0 & I_n
\end{pmatrix} $$
\hspace*{2cm}$\rightarrow M(n)=\mathcal{O}(I(3n))$ \hspace*{4cm} $\Rightarrow M(n) \in \mathcal{O}(I(n))$

Since $I(n) \in \mathcal{O}(n^3)$, $I(3n)\in \mathcal{O}(I(n))$. (Because $n$ is multiplied by $27$)\\

\item $I(n)\in \mathcal{O}(M(n))$?\\

We reduce inversion to "a few" multiplications.

\begin{itemize}
\item  First assume $A=A^T\succcurlyeq 0$ (symetric, positive-definite).\\
\begin{center}
$A=\begin{pmatrix}
B & C^T\\
C & D
\end{pmatrix}$ \hspace*{1cm},\hspace*{1cm}
\begin{tabular}{l}
$B=B^T\succcurlyeq 0$\\
$C=C^T\succcurlyeq 0$
\end{tabular}
\end{center}
$$A^{-1}=\begin{pmatrix}
B^{-1}+B^{-1}C^TS^{-1}CB^{-1} & -B^{-1}C^TS^{-1}\\
-S^{-1}CB^{-1} & S^{-1}
\end{pmatrix}$$
with $S=$ Schur's Complement $=D-CB^{-1}C^T$\\
There are:
\begin{itemize}
\item 2 inversions: $B^{-1}$, $S^{-1}$ of size $\dfrac{n}{2}$
\item 4 multiplications: $CB^{-1}$, $C^T (CB^{-1})$, $S^{-1}(CB^{-1})$, $(CB^{-1})^T\left[S^{-1}(CB^{-1})\right]$
\end{itemize}
\begin{align*}
\Rightarrow I(n) &\leq 2I(\dfrac{n}{2})+4M(\dfrac{n}{2})+\Theta(n^2)&\\
&=2I\left(\dfrac{n}{2}\right)+\Theta(M(n))&\text{(because }n^2\in \mathcal{O}(M(n)))\\
&=\Theta(M(n))&\text{(Master theorem with } a=b=2)\\
\end{align*}
$\Rightarrow I(n)\in \mathcal{O}(M(n))$\\

\item For general invertible matrices:\\
\hspace*{2cm} $A^{-1}=\underbrace{(A^TA)^{-1}}\limits_{\succcurlyeq 0}A^T\rightarrow$we can apply the trick to $(A^TA)$. There is one more multiplication but it does not change anything.
\end{itemize}
\end{enumerate}
\end{proof}

\section{Random Select algorithm}
\textit{How to find the ith smallest entry of an array?}
\begin{example}
\begin{leftbar}
First possible algorithm:
\begin{itemize}
\item $i=1$   minimum: $\Theta(n)$
\item $i=n$   maximum: $\Theta(n)$
\item $i=\lfloor \dfrac{n}{2}\rfloor$ median: $\mathcal{O}(n\log n)$: sort then read $T(\lfloor \dfrac{n}{2}\rfloor)$
\end{itemize}
\end{leftbar}
\end{example}

Can we do better?\\
\underline{Idea}: Do random Quicksort but save effort by not sorting one of the two subarrays.\\

If we make the assumption that all entries are different, we have the following algorithm.

\begin{lstlisting}[label={list:c3:select},caption={Selection (T,i)}]
Selection (T,i) (assumption : all entries are different)
pivot = random entry of T = T(j) for random j in (1,2,...,n)
T_{low} = entries < pivot
T_{hight} = entries > pivot
if |T_{low}| = i-1 then Selection (T,i) := pivot
if |T_{low}| < i-1 then Selection (T,i) := Selection (T_{hight}, i-|t_{low}|-1)
if |T_{low}| > i-1 then Selection (T,i) := Selection (T_{low}, i)
\end{lstlisting}

(Worst-case) expected time$=t_n$
\begin{align*}
t_n &\leq \mathbb{E}[t_{\max(|T_{low}|,|T_{high}|)}]+\Theta (n) & \Theta (n)\rightarrow \text{finding }T_{low},T_{high}\text{ and }|T_{low}|\\
&=\sum\limits_{k=0}^{n-1}P[|T_{low}|=k]t_{\max(k,n-k-1)}+\Theta (n) & \Theta (n)\leq an \text{ (check following theorem)}\\
&=2\sum\limits_{k=\left\lfloor \frac{n-1}{2}\right\rfloor}^{n-1}\dfrac{1}{n}t_k+\Theta (n)&\\
\end{align*}
Clever way to solve it: guess and check if it is correct (many terms otherwise).\\

\begin{theorem}
$t_n=\Theta (n)$
\end{theorem}
\begin{proof}
Assume the term $\Theta (n)\leq an$ for $n$ large enough.\\
Assume $t_n\leq cn$ for some $c>0$ (to be chosen later), all large enough $n$.\\
Proof by induction: assume it's true for $k\leq n-1$: $t_k\leq c_k$.\\
Show that it is true for $n$: $t_n\leq c_n$?
\begin{align*}
t_n &\leq 2\sum\limits_{k=(n-1)/2}^{n-1}\dfrac{t_k}{n}+an\\
&\leq 2\sum\limits_{k=(n-1)/2}^{n-1}\dfrac{ck}{n}+an\\
&=2\dfrac{c}{n}\overbrace{\left[\dfrac{(n-1)(n-2)}{2}-\dfrac{\dfrac{n-1}{2}\left(\dfrac{n-1}{2}-1\right)}{2}\right]}\limits^{\leq \dfrac{3n^2}{8}}+an\\
&=\dfrac{3cn}{4}+a_n
\end{align*}
which is $\leq cn$ if we choose $c$ such that $\dfrac{3c}{4}+a<c \Rightarrow c>4a$
\end{proof}